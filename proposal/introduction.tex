\chapter{Introduction}

The last decade of computer systems research has yielded efficient
kernel-bypass stores with throughput and access latency thousands
of times better than conventional stores.
%
Today,
these systems can execute millions of operations per second
with access
times of 5~\us or less~\cite{mica,ramcloud,farm-2014}.
%
% storage systems have while preserving their performance benefits.
%
These gains come from careful attention to detail in request processing,
so these systems often start with simple and stripped-down designs to
achieve their performance goals.

However, for these low-latency stores to be practical in the long-term,
they must evolve to
include many of the features that conventional data center and cloud
storage systems have while preserving their performance benefits.
%
Key features include the ability to reconfigure and (re)distribute load
(and data) in
response to load imbalances and failures, which occur frequently in
practice;
%
the ability to perform such reconfiguration in the cloud, where
networking stacks have been
historically slow, and where resources are shared by
multiple tenants;
%
and the ability to support a diverse set of such tenants with varying
access patterns, data models and performance requirements.

\textbf{Load Distribution:}
When it comes to load distribution, hash partitioning records across
servers is often the norm since
it is a simple, efficient, and scalable way of distributing load across
a cluster of machines.
%
Most systems tend to pre-partition records and tables into coarse hash
buckets, and then move these buckets around the cluster in response to
load imbalances~\cite{dynamo}.
%
However, coarse pre-partitioning can lead to high request fan-out when
applications exhibit temporal locality in the records they access,
hurting performance and cluster
utilization~\cite{rocksteady}.
%
Therefore, in order to be able to support a diverse set of applications
with different access patterns, these systems need to be more flexible
and lazy about how they partition and distribute data.

Flexible and lazy partitioning creates a unique challenge for
kernel-bypass storage systems.
%
Once a decision to partition is made, the partition must be quickly
moved to it’s new home with minimum impact to performance.
%
Doing so is hard; these
systems offer latencies as low as 5~\us,
so even a few cache misses will significantly hurt
performance.

\textbf{Preserve Performance at Scale:}
Several of these stores exploit many-core hardware to
ingest and index events at high rates -- 100 million operations
(Mops) per second (s) per machine~\cite{mica,flexnic,floem,kvdirect}.
%
However, they rely on application-specific
hardware acceleration, making them impossible to deploy on today's cloud
platforms.
%
Furthermore, these systems only store data in DRAM, and they do not scale across
machines; adding support to do so without cutting into normal-case performance
is not straightforward.
%
For example, many of them statically partition records across cores to
eliminate cross-core synchronization.
%
This optimizes normal-case performance, but it makes concurrent
operations like migration and scale out impossible; transferring record data
and ownership between machines and cores requires a stop-the-world approach
due to these systems' lack of fine-grained synchronization.

Achieving this level of performance while fulfilling all of these
requirements on commodity cloud platforms requires solving two key challenges
simultaneously.
%
First, workloads change over time and cloud VMs fail, so systems must tolerate
failure and reconfiguration.
%
Doing this without hurting normal-case performance at 100~Mops/s is hard, since
even a single extra server-side cache miss to check key ownership or
reconfiguration status would cut throughput by tens-of-millions of operations
per second.
%
Second, the high CPU cost of processing incoming network
packets easily dominates in these workloads,
especially since, historically, cloud networking stacks have not been designed
for high data rates and high efficiency.

\textbf{Expressive Data Model and Multi-tenancy:}
Since the end of Dennard scaling, disaggregation has become the norm in
the datacenter.
%
Applications are typically broken into a compute and
storage tier separated by a high speed network, allowing each tier to be
provisioned, managed, and scaled independently.
%
However, this approach
is beginning to reach its limits.
%
Applications have evolved to become more data intensive than ever.
%
In addition to good performance, they often require rich and complex
data models such as social graphs, decision trees,
vectors~\cite{fb-memcache,parameter-server} etc.
%
Storage systems, on the other hand, have become faster with the help of
kernel-bypass~\cite{ramcloud,farm-txns}, but at the cost of their
interface – typically simple point lookups and updates.
%
As a result of using these simple interfaces to implement their data
model, applications end up stalling on network round-trips to the
storage tier.
%
Since the actual lookup or update takes only a few
microseconds at the storage server, these round-trips create a major
bottleneck, hurting performance and utilization.
%
Therefore, to fully
leverage these fast storage systems, applications will have to reduce
round-trips by pushing compute to them.

Pushing compute to these fast storage systems is
not straightforward.
%
To maximize utilization, these
systems need to be shared by multiple tenants,
but the cost for isolating tenants using conventional techniques is too
high.
%
Hardware isolation
requires a context switch that takes approximately
1.5 microseconds on a modern processor~\cite{splinter}.
%
This
is roughly equal to the amount of time it takes to
fully process an RPC at the storage server, meaning
that conventional isolation can hurt throughput by
a factor of 2.

% Another trend is
% \emph{multi-tenancy}; sharing a resource between many tenants helps improve
% utilization.
%
% However, the granularity (containers and VMs) and density
% (1-2 tenants per core) of sharing in today’s datacenters is insufficient
% to keep these fast storage systems at high utilization.

\section{Contributions}

\noindent{\bf
  Low-latency stores adopt simple, stipped down designs that optimize
  for normal case performance, and in the process, trade off
  features that
  would make them more practical and cost effective at cloud scale.
%
  This thesis shows that this trade off is unnecessary. Carefully
  leveraging and extending new and existing abstractions for scheduling,
  data sharing, lock-freedom, and isolation will yield feature-rich
  systems that
  retain their primary performance benefits at cloud scale.
}

\vspace{8pt}

This thesis presents horizontal and vertical mechanisms for
rapid low-impact reconfiguration, multi-tenancy and expressive data
models that
would help make low-latency storage systems more practical and
efficient at cloud scale.
%
It is structured into three key pieces:

\begin{enumerate}
\item \textbf{Fast Data Migration:}
The first piece presents \emph{Rocksteady}~\cite{rocksteady},
a horizontal mechanism for rapid reconfiguration and elasticity.
%
Rocksteady is a live migration technique for scale-out
in-memory key-value stores.
%
It balances three competing goals: it
migrates data quickly, it minimizes response time impact, and it
allows arbitrary, fine-grained splits.
%
Rocksteady allows a key-value store to defer all repartitioning work
until the moment of migration, giving it precise and timely control for
load balancing.

\item \textbf{Low Cost Coordination:}
The second piece presents \emph{Shadowfax}~\cite{shadowfax},
a system with horizontal and vertical mechanisms that
allow distributed key-value stores to
transparently span DRAM, SSDs, and cloud blob storage while serving
130~Mops/s/VM over commodity Azure VMs using conventional Linux TCP.
%
Beyond
high single-VM performance, Shadowfax uses a unique approach to distributed
reconfiguration that avoids any server-side key ownership checks
or cross-core coordination both during normal operation and migration.

\item \textbf{Extensibility and Multi-Tenancy:}
The final piece presents \emph{Splinter}~\cite{splinter}, a system that
provides clients with a vertical mechanism
to extend low-latency key-value stores by migrating (pushing) code to them.
%
Splinter is designed for
modern multi-tenant data centers; it allows mutually distrusting tenants to write
their own fine-grained extensions and push
them to the store at runtime.
%
The core of
Splinter's design relies on type- and memory-safe
extension code to avoid conventional hardware isolation costs.
%
This
still allows for bare-metal execution, avoids data copying across trust
boundaries, and makes granular storage functions that perform less than
a microsecond of compute practical.

\end{enumerate}

\section{Fast Data Migration}

\emph{Rocksteady} is a live migration technique for scale-out
in-memory key-value stores.
%
Built on top of RAMCloud~\cite{ramcloud}, Rocksteady’s key insight is to
leverage application skew to speed up data migration while minimizing
the impact to performance.
%
When migrating a partition from a source to a target, it first migrates
ownership of the partition.
%
Doing so moves load on the partition from the source to the target, creating
headroom on the source that can be used to migrate data.
%
To keep the partition online, the target pulls records from the source
on-demand; since applications are skewed – most requests are for a small
set of hot records – this on-demand process converges quickly.

To fully utilize created headroom, Rocksteady carefully schedules and
pipelines data migration on both the source and target.
%
Migration is
broken up into tasks that work in parallel over RAMCloud’s hash table;
doing so keeps the pre-fetcher happy, improving cache locality.
%
A shared-memory model allows these tasks to be scheduled on any core,
allowing any idle compute on the source and target to be used for
migration.
%
To further speed up migration, Rocksteady delays
re-replication of migrated data at the target to until after migration
has completed.
%
Fault tolerance is guaranteed by maintaining a dependency
between the source and target at RAMCloud’s coordinator (called lineage)
during the migration, and recovering all data at the source if either
machine crashes.
%
Recovery must also include the target because of early
ownership transfer; the target could have served and replicated writes
on the partition since the migration began.
%
Putting all these parts
together results in a protocol that migrates data 100x faster than the
state-of-the-art while maintaining tail latencies 1000x lower.

Overall, Rocksteady’s careful attention to ownership, scheduling, and
fault tolerance allow it to quickly and safely migrate data with low
impact to performance.
%
Experiments show that it can migrate at 758 MBps
while maintaining tail latency below 250 microseconds; this is
equivalent to migrating 256 GB of data in 6 minutes, allowing for quick
scale-up and scale-down of a cluster.
%
Additionally, early ownership transfer and
lineage help improve migration speed by 25\%.
%
These results have
important implications on system design; fast storage systems can use
Rocksteady as a mechanism to enable flexible, lazy partitioning of
data.

\section{Low Cost Coordination}

\emph{Shadowfax} allows distributed key-value store to
transparently span DRAM, SSDs, and cloud blob storage.
%
Its unique approach to
distributed reconfiguration avoids any cross-core coordination during
normal operation and data migration both in its indexing and network interactions.
%
In contrast to totally-ordered or stop-the-world approaches used by most
systems, cores in Shadowfax avoid stalling to synchronize with one another, even when
triggering complex operations like scale-out, which require
defining clear before/after points in time among concurrent operations.
%
Instead, each core participating in these operations -- both at clients and
servers -- independently decides a point in an \emph{asynchronous global
cut} that defines a boundary between operation sequences in these complex operations.
%
Shadowfax vertically extends asynchronous cuts from cores within one
process~\cite{faster} to servers
and clients in a cluster.
%
This helps coordinate server
and client threads
in Shadowfax's low-coordination data migration and
reconfiguration protocol.

In addition to reconfiguration, Shadowfax has mechanisms that help it
achieve a high throughput of 130~Mops/s/VM over
commodity Azure VMs~\cite{azure}.
%
First, all requests from a client on one machine to Shadowfax are asynchronous with
respect to one another all the way throughout Shadowfax's client- and
server-side network submission/completion paths and servers' indexing and
(SSD and cloud storage) I/O paths.
%
This avoids all client- and server-side stalls due to head-of-line
blocking, ensuring that clients can always continue to generate requests and
servers can always continue to process them.
%
Second, instead of partitioning data among cores to avoid synchronization on record
accesses~\cite{hstore,voltdb,mica,seastar}, Shadowfax partitions network
sessions across cores; its lock-free hash index and log-structured record heap
are shared among all cores.
%
This risks contention when some records are hot and frequently
mutated, but this is more than offset by the fact that no software-level
inter-core request forwarding or routing is needed within server VMs.

Measurements show that Shadowfax can shift load in 17~s to improve system throughput by
10~Mops/s
with little disruption. Compared to the state-of-the-art, it has 8$\times{}$ better throughput
  (than Seastar+memcached) and scales out 6$\times{}$ faster.
%
When scaled to a small cluster, Shadowfax retains its high throughput to
perform 400~Mops/s,
%
which, to the best of our knowledge, is the highest
reported throughput for a distributed key-value store used for
large-scale data ingestion and indexing.

\section{Extensibility and Multi-Tenancy}

\emph{Splinter} provides clients with a vertical mechanism
to extend low-latency key-value stores by migrating (pushing) code to them.
%
Splinter relies on a type- and memory-safe language for isolation.
%
Tenants push
extensions – a tree traversal for example – written in the Rust
programming language~\cite{rust} to the system at runtime.
%
Splinter installs
these extensions.
%
Once installed, an extension can
be remotely invoked (executed) by the tenant in a
single round-trip.
%
For applications such as tree traversals which would ordinarily require
round-trips logarithmic in the size of the tree, splinter can
significantly improve both throughput and latency.

In addition to lightweight isolation, splinter consists of multiple
mechanisms to make pushing compute feasible.
%
Cross-core synchronization
is minimized by maintaining \emph{tenant locality}; tenant requests are
routed
to preferred cores at the NIC~\cite{flow-director} itself, and cores
steal work from
their neighbour to combat any resulting load imbalances.
%
Pushed code (an extension) is scheduled \emph{cooperatively}; extensions are
expected to yield down to the storage layer frequently ensuring that
long running extensions do not starve out short running ones.
%
This
approach is preferred over conventional multitasking using kthreads
because preempting a kthread requires a context switch, making it too
expensive for microsecond timescales.
%
Uncooperative extensions are
identified and dealt with by a dedicated watchdog core.
%
Data copies are
minimized by passing immutable references to extensions; the rust
compiler statically verifies the lifetime and safety of these
references.
%
With the help of these mechanisms, Splinter can isolate
100’s of granular tenant extensions per core while serving millions of
operations per second with microsecond latencies.

Overall, Splinter adds extensibility to fast kernel-bypass storage
systems, making it easier for applications to use them.
%
An 800 line Splinter extension implementing Facebook’s TAO graph
model~\cite{tao-2013}
can serve 2.8 million ops/s on 8 threads with an average latency of
30 microseconds.
%
A significant fraction of TAO operations involve only a single
round-trip.
%
Implementing these on the client using normal lookups and
implementing the remaining operations using the extension helps improve
performance to 3.2 million ops/s at the same latency.
%
This means that an
approach that combines normal lookups/updates with Splinter’s extensions
is the best for performance; the normal lookups do not incur isolation
overhead (no matter how low), and the extensions reduce the number of
round-trips.
%
In comparison, FaRM’s~\cite{farm-2014} implementation of TAO performs
6.3 million
ops/s on 32 threads with an average latency of 41 microseconds.
%
This
makes Splinter’s approach, which performs 0.4 million ops/s per thread,
competitive with FaRM’s RDMA based approach, which performs 0.2 million
ops/s per thread.
