\chapter{Extensibility and Multi-Tenancy}

Since the end of Dennard scaling, disaggregation has become the norm in
the datacenter.
%
Applications are typically broken into a compute and
storage tier separated by a high speed network, allowing each tier to be
provisioned, managed, and scaled independently.
%
However, this approach
is beginning to reach its limits.
%
Applications have evolved to become more data intensive than ever.
%
In addition to good performance, they often require rich and complex
data models such as social graphs, decision trees,
vectors~\cite{fb-memcache, parameter-server} etc.
%
Storage systems, on the other hand, have become faster with the help of
kernel-bypass~\cite{ramcloud, farm-txns}, but at the cost of their
interface – typically simple point lookups and updates.
%
As a result of using these simple interfaces to implement their data
model, applications end up stalling on network round-trips to the
storage tier.
%
Since the actual lookup or update takes only a few
microseconds at the storage server, these round-trips create a major
bottleneck, hurting performance and utilization.
%
Therefore, to fully
leverage these fast storage systems, applications will have to reduce
round-trips by pushing compute to them.

Pushing compute to these fast storage systems is
not straightforward.
%
To maximize utilization, these
systems need to be shared by multiple tenants,
but the cost for isolating tenants using conventional techniques is too
high.
%
Hardware isolation
requires a context switch that takes approximately
1.5 microseconds on a modern processor~\cite{splinter}.
%
This
is roughly equal to the amount of time it takes to
fully process an RPC at the storage server, meaning
that conventional isolation can hurt throughput by
a factor of 2 (Fig 1).
