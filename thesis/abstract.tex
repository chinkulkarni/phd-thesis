The last decade of computer systems research has yielded efficient
kernel-bypass stores with throughput and access latency thousands
of times better than conventional stores.
%
These gains come from careful attention to detail in request processing,
so these systems often start with simple and stripped-down designs to
achieve their performance goals.
%
Hence, they trade off
features that
would make them more practical and cost effective at cloud scale,
such as load (re)distribution, multi-tenancy, and
expressive data models.

This thesis demonstrates that this trade off is unnecessary.
It presents mechanisms for
reconfiguration, multi-tenancy and expressive data
models that
would make these systems more practical and
efficient at cloud scale while preserving their performance
benefits.

\emph{Rocksteady}
is a live migration technique for scale-out
in-memory key-value stores.
%
It balances three competing goals: it
migrates data quickly, it minimizes response time impact, and it
allows arbitrary, fine-grained splits.
%
Rocksteady allows a key-value store to defer all repartitioning work
until the moment of migration, giving it precise and timely control for
load balancing.

\emph{Shadowfax} is
a system that
allows distributed key-value stores to
transparently span DRAM, SSDs, and cloud blob storage while serving
130~Mops/s/VM over commodity Azure VMs using conventional Linux TCP.
%
Beyond
high performance, Shadowfax uses a unique approach to distributed
reconfiguration that avoids any server-side key ownership checks
or cross-core coordination both during normal operation and migration.

\emph{Splinter} is a system that
allows clients
to extend low-latency key-value stores by migrating (pushing) code to them.
%
Splinter is designed for
modern multi-tenant data centers; it allows mutually distrusting tenants to write
their own fine-grained extensions and push
them to the store at runtime.
%
The core of
Splinter's design relies on type- and memory-safe
extension code to avoid conventional hardware isolation costs.
%
This
still allows for bare-metal execution, avoids data copying across trust
boundaries, and makes granular storage functions that perform less than
a microsecond of compute practical.
