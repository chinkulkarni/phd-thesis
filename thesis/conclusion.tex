\chapter{Conclusion}

Low-latency stores adopt simple, stripped-down designs that optimize
for normal-case performance, and in the process, trade-off
practicality and cost-effectiveness at cloud scale.

This dissertation shows that this trade-off is unnecessary. Carefully
leveraging and extending new and existing abstractions for scheduling,
data sharing, lock-freedom, and isolation will yield feature-rich
systems that
retain their primary performance benefits at cloud scale.

Rocksteady is a migration protocol for in-memory key-value stores that avoids
the need for and overhead of in-advance state partitioning; it eliminates
replication overhead from the migration fast-path; it exploits parallelism;
it exploits modern NIC hardware.  Rocksteady has a
``pay-as-you-go'' approach that helps avoid overloading the source during migration
using asynchronous batched on-demand pulls to shift load away from the source
as parallel background transfers proceed.  In all, Rocksteady can move the entire DRAM
of a modern data center machine in a few minutes while retaining \nnnth
percentile tail latency of lesser than 250~\us.

Shadowfax allows distributed key-value stores to
span DRAM, SSDs, and cloud blob storage transparently.
%
Its unique approach avoids cross-core coordination during
regular operation and data migration both in its indexing and network
interactions.
%
Instead of partitioning data among cores to avoid synchronization on record
accesses, Shadowfax partitions network
sessions across cores and shares its lock-free hash index and log-structured
record heap among them.
%
This risks contention when some records are hot and frequently
mutated, but this is more than offset by avoiding software-level
intercore request forwarding or routing within server VMs.
%
In contrast to totally-ordered or stop-the-world approaches used by most
systems, cores in Shadowfax avoid stalling to synchronize with one another, even when
triggering complex operations like scale-out, which require
defining clear before/after points in time among concurrent operations.
%
Instead, each core participating in these operations -- both at clients and
servers -- independently decides a point in an asynchronous global
cut that defines a boundary between operation sequences in these complex operations.
%
Compared to the state-of-the-art, it has 8$\times{}$ better throughput
and scales out 6$\times{}$ faster.

Splinter shows that soon legacy hardware isolation techniques will
  limit resource provisioning granularity in the cloud, but it also provides a
  way forward.
Systems must evolve to support granular, low-overhead shipping of compute to
    storage, and lightweight isolation between small compute tasks.
Splinter works toward that evolution by discarding hardware isolation in
  favor of static safety checks.
As a result, it supports thousands of tenants that can all access data in tens of
  microseconds while customizing storage operations to their needs and while
  performing millions of remote operations on modern multicore machines.
