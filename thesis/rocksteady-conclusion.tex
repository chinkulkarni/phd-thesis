\section{Conclusion}

Low-latency in-memory stores are designed to tolerate the heaviest request
loads, but if they are too stripped down they cannot deal with complex
higher-level operations like reacting to workload changes, skew shifts, and load spikes.
Rocksteady is a migration protocol for in-memory key-value stores that avoids
the need for and overhead of in-advance state partitioning; it eliminates
replication overhead from the migration fast path; it exploits parallelism; and
it exploits modern NIC hardware.  Rocksteady has a
``pay-as-you-go'' approach that helps avoid overloading the source during migration
using asynchronous batched on-demand pulls to shift load away from the source
as parallel background transfers proceed.  In all, Rocksteady can move the entire DRAM
of a modern data center machine in a few minutes while retaining \nnnth
percentile tail latency of less than 250~\us.
