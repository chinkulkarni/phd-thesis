\section{Discussion} 

%\subsection{Lessons for Other Systems}

Some of the most broadly applicable lessons from Rocksteady are on the
interplay of partitioning, dispatch, and synchronization. Recent works have
often partitioned operations~\cite{hstore,redis} or sometimes just mutating
operations~\cite{mica} to reduce locking and contention.  Systems that strictly
partition work (even only writes) are likely to have to reconfigure more often
under skew.  Their access latencies also suffer since migration must be
interleaved with regular execution.  RAMCloud's dispatch can be a bottleneck,
but it can also redirect any idle CPU resources on a few microseconds timescale,
which is key to Rocksteady's adaptive parallel replay and tight SLAs.
Hardware-assisted~\cite{flexnic}, client-assisted~\cite{mica}, and parallel dispatch help
mitigate bottlenecks and delay the need for migration. However, none of these can
eliminate the need for cross-machine rebalancing or the need to overlap normal
execution and migration.
%
Optimizing for normal-case request processing can make inevitable
background system tasks more costly.  Designers of in-memory systems must
carefully navigate partitioning, dispatch, and locking trade-offs when planning
for heavy rebalancing operations, like migration.

Rocksteady's safe deferred re-replication can also be applied to other systems.
For example, H-Store with the Squall~\cite{squall} migration system could
exploit the same idea to improve migration throughput and access distribution
impact.  Squall could take a temporary dependency on source data and backups to
avoid synchronous re-replication at the target; this would have a significant
impact since re-replication blocks execution on the whole target partition in
Squall.

%\subsection{Lessons in Fast Dispatch}
%
%% Maybe covered above now.
%
%Over optimizing dispatch has perils. RAMCloud's simplistic dispatch model is
%one of its key limitations, but it also helps minimizing impact of migration
%operations.
%
%Data migration demonstrates the effectiveness of RAMCloud's dispatch model.
%
%
%
%Other possible points
%\begin{itemize}
%  \item
%  Can use all this to massive accelerate RAMCloud's recovery and improve its
%  scalability.
%\end{itemize}

\subsection{Going Even Faster}

Rocksteady can migrate hundreds of megabytes per second with tight response
latency, but it still only uses a small fraction of the bandwidth provided by
modern networks. While its approach and implementation can be tuned for
some gains, it is unlikely that simple changes would result in the
order-of-magnitude speed up that would be needed to saturate the network.

To achieve such gains without destroying normal-case request processing,
migration might be limited to merely transferring large, opaque memory regions
between hosts, with little-to-no packaging or replay work on either end. This
approach would require the source to physically partition its state in fine
enough units to satisfy all possible future splits. FaRM's data
layout, for example, meets these properties~\cite{farm-txns}.

Physically partitioning groups of records on key or key hash would constrain
RAMCloud's log-structured memory cleaning. The cleaner minimizes cleaning CPU
and memory bandwidth load by physically colocating records that are likely to
have a similar lifetime~\cite{ramcloud-lsm}.  With physical partitioning constraints, the cleaner
would not be able to optimize the hot/cold separation of objects globally.
Investigating the cleaner's sensitivity to such partitioning could be an
interesting direction since it might be able to assist in the
process of physically partitioning records.

Even if records were partitioned and could be moved at line-rate, it is possible
that RAMCloud would need network-level support in order to avoid interference
between large, fast migration transfers and fine-grained normal-case requests.

Beyond improvements in dispatch scalability, other improvements to
RAMCloud's concurrency model could also have a significant impact on
Rocksteady.  Today, RAMCloud processes requests on workers that use standard
kernel threads.  Coroutines or cooperative user-level threading could both
improve response distributions and efficiency~\cite{fasst-2016}. If \pull and replay operations
could afford frequent yields to RAMCloud's dispatch, heavy operations would
have less impact on normal-case request processing. Replay and \pull operations
could be coarser as well, resulting in fewer requests and lower dispatch
overheads. This optimization could allow Rocksteady to transfer data even more quickly with
the same SLAs.

%Discuss tradeoffs of approach that breaks tablets 10,000 ways before migration
%(probably not all at once)?
