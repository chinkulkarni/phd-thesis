\section{Related Work}
\label{sec:related}

Shadowfax builds on several areas of recent
research.

\subsection{Epochs and Cuts}
There are many schemes for synchronization and memory protection in lock-free
concurrent data structures including hazard pointers~\cite{hazard},
read-copy-update~\cite{rcu} and epoch-based schemes~\cite{epochs,epochs-phd}.
Like Shadowfax, several other
systems~\cite{bwtree,deuteronomy,deuteronomy-ranges,hekaton-indexing} use epochs for this purpose.

% Notably, many of these systems combine epoch protection with copy-on-write or
% log-structured updates, which adds overhead due to allocation and tail-of-log
% contention. Other systems use update-in-place with latching~\cite{masstree} to avoid this overhead, but target main memory. Shadowfax
% uses update-in-place for values in memory, and switches to copy-on-write via a lock-free
% hash index when values are stable; its unique approach to epochs is key to avoiding overhead
% when determining which values require copy-on-write.

Shadowfax's use of epochs resembles Silo's, a (single-node)
in-memory store~\cite{silo}. Like in Silo, Shadowfax's epochs avoid strong
ordering among requests except on coarse boundaries, improving scalability.
Silo also uses epochs to improve write-ahead logging
scalability~\cite{silor}.  Shadowfax extends these epochs back to clients by
asynchronously choosing points in server execution and correlating these back
to per-client sequence numbers, effectively pushing the overhead of logging out
of servers altogether. Similarly, Scalog's persistence-before-ordering approach uses
global cuts that define and order shards of operations on different
machines~\cite{scalog}; Shadowfax uses similar cuts across threads and client
session buffers to define an order to enforce boundaries among operations.

\subsection{High-throughput Networked Stores}
Some in-memory stores exploit kernel-bypass networking or RDMA and optimize
for multicore. Many of these focus on throughput but do not provide
scale out~\cite{pilaf,mica,herd}, both of which
can slow normal-case request processing. RAMCloud focuses on low-latency
and has migration, but its throughput is two
orders of magnitude less than Shadowfax~\cite{ramcloud,ramcloud-recovery}.

% Anna~\cite{Wu:2019:ATC:3311880.3322438} uses selective key replication with
% a per-thread shared nothing design, but relies on eventual consistency with
% lazy gossip.

FaRM~\cite{farm-2014,farm-txns} creates large clusters of distributed memory where
clients primarily use one-sided RDMA reads to construct data structures like
hash tables. FaRM supports scale out and crash recovery by relying on
whole-machine battery backup and in-memory replication.  FaRM's reported
per-core throughput is about 300,000 reads/s/core, compared to Shadowfax's
1.5~million read-modify-writes/s/core, though there are differences in
experimental set up. For example, FaRM doesn't report numbers for
read-modify-write or write-only workloads which are significantly more
expensive in FaRM, since they involve server CPU, require replication, and
cannot be done with one-sided RDMA operations.

% Elasticity

\subsection{Elasticity}
Scale out and migration are key features in shared,
replicated stores~\cite{dynamo,cassandra,redis}.  High-throughput, multicore
stores complicate this because normal-case request processing is highly
optimized and migration competes for CPU.  Some stores rely on
in-memory replicas for fast load redistribution~\cite{farm-txns,drtmb};
this is expensive due to DRAM's high cost and
replication overhead.
% , so this does not work for Shadowfax.

Squall~\cite{squall} migrates data in the H-Store~\cite{hstore}
database; it exploits skew via on-demand record pulls from
source to target with colder data moved in the background.
Rocksteady~\cite{rocksteady} uses this idea in RAMCloud along with a
deferred replication scheme that avoids write-ahead logging for
migrated data.
