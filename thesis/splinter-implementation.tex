\section{Implementation}
\label{sec:implementation}

The Splinter store is implemented in 7,500 lines of Rust. It uses the
  NetBricks network function virtualization framework~\cite{netbricks-2016} as a
  wrapper over the DPDK~\cite{dpdk} packet processing framework.
Splinter also includes 1,100 lines of Rust that
  provide the store interface to extensions. Extensions import it and
  compile against it.
The store also imports the interface, since it defines how the store interacts
  with extensions to create a new generator for an invocation.
The Splinter codebase is open sourced and freely available on github at
the following link:
\url{https://github.com/utah-scs/splinter}.

The store need not be written in Rust, but doing so has advantages.
It prevents data races and segmentation faults within the store, but
  it also lets the store use Rust's type system and lifetimes to ensure that
  mistakes are not made with lifetimes of objects and references
  handed across trust boundaries, which an adversary could exploit.
